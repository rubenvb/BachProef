\documentclass[a4paper,11pt]{article}
\usepackage{a4wide}
\usepackage{geometry}
  \geometry{paper=a4paper,top=2.2cm,bottom=2.9cm}
   \geometry{right=2.3cm,left=2.3cm}
% Feynman diagrammen
\usepackage{feynmp}
\unitlength = 1mm
% Afbeeldingen
\usepackage{graphicx} % afbeeldingen
\usepackage{subfigure} % pakket voor het gebruiken van sub-figuren
\usepackage{float} % laat toe [H] te gebruiken bij plaatsen figuren
% Taal
\usepackage[dutch]{babel}
%\hyphenation{}
% De formules
\usepackage{amssymb, amsmath} % wiskunde
\numberwithin{equation}{section} % (#.#) i.p.v. (#)
% xetex
\usepackage{fontspec}
\usepackage{xunicode}
\usepackage{xltxtra}
\setromanfont[Mapping=tex-text]{Calibri}
% appendix
\usepackage{appendix}


\title{Dwarse energiestroom in de LHeC met saturatie}
\author{Ruben Van Boxem}

\begin{document}

% titelpagina
\fontsize{12pt}{14pt}\selectfont

\begin{center}

\includegraphics[height=3cm]{Afbeeldingen/UA.eps}

\vspace{1cm}

\fontsize{14pt}{17pt}\selectfont
% De Faculteit:
\textsc{Faculteit Wetenschappen} \\
\textsc{Departement Fysica}
\fontsize{12pt}{14pt}\selectfont
\vspace{0.3cm}

\vspace{1.2cm}

%Het academiejaar: aanpassen!
Academiejaar 2009--2010

\vspace{2.8cm}

\fontsize{17.28pt}{21pt}\selectfont

% De titel van de thesis:
\textsc{Dwarse energiestroom in de LHeC met saturatie}

\fontseries{m}
\fontsize{12pt}{14pt}\selectfont

\vspace{3cm}

% De auteur van de thesis:
Ruben \textsc{Van Boxem}	


\vspace{2cm}

Promotor: Prof. Dr Pierre \textsc{Van Mechelen}\\
Copromotor: Krzysztof \textsc{Kutak} \\
\vspace{2cm}
\end{center}
% De functie van de thesis:
Proefschrift voorgedragen tot \\
het behalen van de graad van\\
\textsc{Bachelor in de Fysica}


\thispagestyle{empty}
\newpage

\section*{Dankwoord}
TODO
\thispagestyle{empty}
\newpage
\fontsize{11pt}{14pt}\selectfont

\section*{Summary}
TODO
\thispagestyle{empty}
\newpage

\tableofcontents
\thispagestyle{empty}
\newpage

\section{Diepe inelastische verstrooiing}

Om de diepere structuur van een object zoals het proton te bestuderen, kan men elektron verstrooiing gebruiken.
Door middel van een elektron met hoge energie op een proton te laten botsen kan men de onderliggende samenstelling van het proton bestuderen.
Een proces dat zich hier goed voor leent, heet DIS, oftewel \textit{Deep Inelastic Scattering}. In figuur \ref{fig:DIS} wordt een DIS proces schematisch weergegeven.
\begin{figure} [H]
  \begin{center}
    %\includegraphics[width=.66\textwidth]{Afbeeldingen/...}
    \caption{Elektron-proton verstrooiing. Het elektron zendt een foton uit dat het proton ``onderzoekt".
$W$ is de invariante massa van het uitgaande systeem (dus zonder elektron), $q$ is de overgedragen vierimpuls, $x$ (Björken x) is de fractie van de totale impuls van het proton gedragen door de beschouwde quark.}
   \label{fig:DIS}
  \end{center}
\end{figure}
 Men spreekt van DIS processen als $Q^2 \gg M^2$ (``diep") en $W^2 \gg M^2$ (``inelastisch").
Omdat de waarde van $q^2$ negatief is, wordt vaak volgende variabele ingevoerd: $Q^2 :=-q^2 > 0$. De golflengte van het het foton is van de grootteorde van $1/Q$.
Dit wil concreet zeggen dat als de overgedragen impuls verhoogt, de golflengte van het foton en dus de grootte van de deeltjes waarmee het interageert, kleiner wordt.
In wat volgt wordt beschreven wat er gebeurt als men $Q^2$ systematisch verhoogt.

  \subsection{Breking van schaalinvariantie}
    \paragraph{Elektron-kern verstrooiing}
Stel dat een elektron verstrooid wordt op een atoomkern (bestaande uit meerdere nucleonen).
Als $\lambda \gg d_{kern}$, ziet het foton de kern als een puntdeeltje en wordt het elastisch verstrooid op een puntdeeltje met de lading van de kern.
Als $\lambda \ll d_{kern}$, dringt het foton diep door in de kern en kan het verstrooid worden op een proton in de kern.
Men spreekt hier van diep inelastische elektron-kern verstrooiing, en men kan het volgende schrijven:
\begin{align}
x_N = \frac{M}{M_N} \left( \frac{Q^2}{2M\nu} \right) = \frac{1}{A}
\end{align}
waarbij $M_N$ de massa van de kern is, $M$ de massa van het proton, $\nu$ de frequentie van het foton en $A$ het aantal nucleonen in de kern.
Als men de werkzame doorsnede van deze processen zou uitzetten t.o.v. $x_N$, ziet men een duidelijke verschil tussen de gevallen waar $\lambda \gg d_{kern}$ en die waar $\lambda \ll d_{kern}$.
Dit fenomeen heet schaalinvariantie, waarbij de waargenomen werkzame doorsnede blijkbaar afhangt van de grootte van de schaal, $Q^2$.

      \paragraph{Elektron-proton verstrooiing}
Zolang dat $\lambda \gg d_{proton}$, ziet het foton het proton als puntdeeltje. Het wordt elastisch verstrooid door het proton.
Als $Q^2$ zo groot is dat $\lambda \ll d_{proton}$, wordt er opnieuw ``ingezoomd", waarbij het foton nu de bouwstenen van het proton kan ``zien" (i.e. ermee kan interageren).
Het foton ziet dus de 3 quarks als puntdeeltjes en men spreekt van diep inelastische elektron-proton verstrooiing.
Kinematisch geldt het volgende (zie appendix \ref{app:DIS}):
\begin{align} \label{eq:Bjorkenx}
x = \left( \frac{Q^2}{2p\cdot q} \right)
\end{align}
$x$ wordt aangeduid als de ``Bjorken schaalvariabele".
De waarschijnlijkheid van zo'n proces worden dus enkel en alleen bepaalt door $x$ en niet $Q^2$ en $p\cdot q$ afzonderlijk.

      \paragraph{Breking van schaalinvariantie}
In volgorde van stijgende $Q^2$ heeft men dus voor de werkzame doorsnede:
\begin{itemize}
  \item ``kern"schaling met een piek op $x_N=1$
  \item  breking van schaalinvariantie
  \item ``proton"schaling met een piek \textit{rond} $x \sim 1$
  \item breking van schaalinvariantie
  \item ``Bjorken"schaling met een \textit{uitgesmeerde} piek rond $x \sim 1/3$ (3 quarks)
\end{itemize}
Als men $Q^2$ blijft vergroten, verwacht men logischerwijs nogmaals een breking van schaalinvariantie te zien, en deze worden ook geobserveerd.
De schijnbare breking van schaalinvariantie ligt in lijn met de voorspellingen van QCD, de theorie van de sterke wisselwerking.

  \subsection{zeequarks, gluonen en $\alpha_S$}
Quarks blijken niet uit een diepere structuur te bestaan (zodat er niet opnieuw een breking van schaalinvariantie optreedt).
In plaats van een diepere quarkstructuur, beschrijft QCD het proton als 3 valentiequarks (de klassieke $uud$ combinatie) plus een willekeurig aantal zeequarks ($q\bar{q}$ paren).
Het foton ziet boven drie puntdeeltjes ook nog een zee van quarks.
Zeequarks ontstaan uit gluonen ($g\rightarrow q\bar{q}$) die zelf uitgestraald worden door andere quarks ($q\rightarrow gq$).
Er treedt dus breking van schaalinvariantie op omdat elk parton (quark of gluon) 



  \subsection{blad}

\section{Dwarse energiestroom}
  \subsection{Formulering}
  \subsection{Gluon dichtheid}
  \subsection{$F_2$}
  \subsection{$\mathcal{F}_2$}
  \subsection{Saturatie}

\section{Resultaten}

\section{Bespreking}

\section{Besluit}

\newpage
\appendix
\section{Diep Inelastische Verstrooiing} \label{app:DIS}
Als men figuur \ref{fig:DIS} beschouwt in het IMF ($m_e \approx m_p \approx 0$), geldt wegens behoud van energie:
\begin{align}
&(xP+q)^2 = m_e = 0 \notag \\
\Leftrightarrow &x^2 P^2 + 2xPq + q^2 = 0 \notag \\
\Leftrightarrow &2xPq -Q^2=0
\end{align}
Waarbij gebruikt is dat $-q^2 = Q^2 \gg x^2P^2$ in het  IMF. Dit laatste is equivalent met \eqref{eq:Bjorkenx}.






\end{document}